%! TEX root = ../main.tex

\chapter{Additional Content}
\label{app:other}

\section{Approximating the upper limit of the UV irradiance of an EAS}
\label{app:cr-uv-irradiance}

In the following, we estimate the UV irradiance, $I_\text{UV}$, or Wattage 
deposited per area, from fluorescence light stemming from an extensive air 
shower. To illustrate that telescopes need to be incredibly sensitive to 
observe this phenomenon, we assume fantastic to unrealistically good conditions
for the \UV light yield and related parameters. This thus results in a very 
optimistic upper limit to $I_\text{UV}$.

Consider a cosmic ray with energy $E_0 = \SI{e20}{\eV}$ impinging almost 
vertically ($\theta=\SI{15}{\degree}$\todo{make upright}) on the upper atmosphere.
We assume the behaviour of the particle cascade resulting from the deeply 
inelastic scattering processes of the primary particle is completely determined 
by the Heitler-Matthews model (see \todo{write this}). We then arrive at the 
following expression for the atmospheric depth (height), at which the \EAS 
reaches its maximum in multiplicity:

\begin{equation}
\label{eq:irradiance-multiplicity}
\Xmax = \SI{600}{\gram\per\centi\meter\squared}\qquad\left(\;\widehat{\approx}\;\SI{20}{\kilo\meter}\;\text{above earth surface} \right).
\end{equation}



of the primary particle behaves purely Heitlerian, we arrive at
the following value for the atmospheric depth (height) at which the shower 
reaches its maximum in multiplicity:

$\text{N}_2$ is the 2P(0,0)
transition \cite{Ave2008}, which lies at $\lambda=\SI{337.1}{\nano\meter}$.
