%! TEX root = ../main.tex

\chapter{Additional Content}
\label{app:other}

\section{Approximating the upper limit of the UV irradiance of an EAS}
\label{app:cr-uv-irradiance}

In the following, we estimate the UV irradiance, $I_\text{UV}$, or Wattage 
deposited per area, from fluorescence light stemming from an extensive air 
shower. To illustrate that telescopes need to be incredibly sensitive to 
observe this phenomenon, we assume fantastic to unrealistically good conditions
for the \UV light yield and related parameters. This thus results in a very 
optimistic upper limit to $I_\text{UV}$.

Consider a cosmic ray with energy $E_0 = \SI{e18}{\eV}$ impinging almost 
vertically ($\theta=\SI{15}{\degree}$\todo{make upright}) on the upper 
atmosphere. We assume the behaviour of the particle cascade resulting from the 
deeply inelastic scattering processes of the primary particle is completely 
determined by the Heitler-Matthews model (see \todo{write this} and 
\cite{Matthews2005, Risse2006}). We then arrive at the following value for the 
atmospheric depth (height), at which the \EAS reaches its maximum in 
multiplicity:

\begin{equation}
\label{eq:irradiance-multiplicity}
\Xmax \approx \SI{700}{\gram\per\centi\meter\squared}\qquad\left(\;\widehat{\approx}\;\SI{18.5}{\kilo\meter}\;\text{above earth surface}\footnotemark\right).
\end{equation}
\footnotetext{The calculated height above surface that corresponds to the given 
atmospheric depth varies a lot based on atmospheric variables. We have assumed a
purely isothermic atmosphere with $T=\SI{278}{\kelvin}$, and a standard density 
of $\rho=\SI{0.86}{\kilogram\per\meter\cubed}$ at an altitude of 
\SI{1400}{\meter} above sea level.}

At $\Xmax$, the EM component of the shower contains roughly
$1-\left(E_0\,/\,\xi^\pi_C\right)^{\beta-1}\approx92\%$ of the primary particle 
energy, with photons and electrons sharing the fraction to equal parts.
Continuing, we assume that only electrons contribute meaningfully to excitations
of air molecules. Muons in the hadronic component of the \EAS are minimally 
ionizing and can therefore be neglected, similar arguments apply for photons in 
the EM component.

It thus follows that $E_\text{UV}=0.92\cdot0.5\cdot\SI{e18}{\electronvolt}=
\SI{4.6e17}{\electronvolt}$ are available a priori to create fluorescence light.
As per \cite{Keilhauer2013}, optimistic numbers for the fluorescence light yield
are about $\text{FY}=8\gamma\,/\,\SI{}{\mega\electronvolt}$. Most of these 
photons stem from the 2P(0,0) transition of N$_2$ \cite{Ave2008}, which has a 
characteristic wavelength of \SI{337.1}{\nano\meter}, and an average radiation 
time of \SI{42}{\nano\second} \cite{Lean1989}. 

Assuming that all available energy $E_\text{UV}$ is immediately converted to 
molecular excitations of N$_2$ at $\Xmax$, and then gradually released according
to an exponential decay, we recover an expression for the (UV) power output of 
the air shower.

\begin{equation}
\label{eq:irradiance-power}
P(t)= \text{FY}\,E_\text{UV}\,\frac{hc}{\SI{337.1}{\nano\meter}} \cdot\frac{e^{-\SI{42}{\nano\second}/t}}{t}   
\end{equation}

For the example at hand, this results in a maximum power output of 
$P_\text{max}\approx\SI{19.4}{\watt}$. This macroscopic number might seem
optimistic for such a microscopic process, but in reality is not surprising, as 
the shower itself is incredibly short lived. 

In a last step, we convert the maximum power output to an irradiance measured on
ground. For this, we must take into acount that the area illuminated by the 
\UV-light grows proportional to the distance squared to the source (location of 
$\Xmax$). Furthermore, \UV-light is attenuated by the atmosphere. More 
specifically, UV light with a wavelength of \SI{337}{\nano\meter} drops in 
intensity by about 13\%/km in clear conditions 
\cite[see Fig. 83 on page 103 of][]{Baum1950}. Supposing that an observer is at
least $d=\SI{18.5}{\kilo\meter}$ away from $\Xmax$ (compare 
\cref{eq:irradiance-multiplicity}), we estimate the total irradiance this 
observer measures as

\begin{equation}
\label{eq:irradiance}
I_\text{UV} = P_\text{max}\,\frac{e^{-0.14\,\frac{d}{\SI{}{\kilo\meter}}}}{4\pi\,d^2}\approx\SI{0.32}{\nano\watt\per\square\meter}.
\end{equation}
