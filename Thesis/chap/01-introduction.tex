\chapter{Introduction}
\label{chap:introduction}

\blindtext