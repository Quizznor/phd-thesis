%! TEX root = ../main.tex

\chapter[The Pierre Auger Observatory]{The Pierre Auger Observatory}
\label{chap:pierre-auger-observatory}

The \PAO is the (by area) largest scientific experiment in the world. It 
consists of an array of 1660 \WCDs, which form the \SD, and 27 fluorescence 
telescopes, that make up the \FD.

With a region spanning roughly \SI{3000}{\kilo\meter\squared} in the Argentinian
pampa at a median elevation of \SI{1400}{\meter}, it offers a unique possibility
to observe \UHECRs at the tail-end of the \CR energy spectrum. 

We begin this chapter in \cref{sec:science-case} by stating the design 
intentions and formulating open questions that the \PAO aims to answer. Design
details for \FD and \SD are given in \cref{sec:fd} and \cref{sec:sd} 
respectively. Following a discussion on the \DAQ process and event detection in 
\cref{sec:cdas}, we finish by detailing the event reconstruction and higher 
level analysis in \cref{sec:rec}.

\section{Science Goal and Open Questions}
\label{sec:science-case}

\blindtext

\subsection{Mass Composition}

\blindtext

\subsection{Muon Deficit}

\blindtext

\subsection{AugerPrime}

\blindtext

\section{The Fluoresence Detector}
\label{sec:fd}

\blindtext

\section{The Surface Detector}
\label{sec:sd}

\blindtext

\section{Central Data Acquisition System}
\label{sec:cdas}

\blindtext

\section{\Offline and Event Reconstruction}
\label{sec:rec}

\blindtext
