%! TEX root = ../main.tex

\chapter{The Pierre Auger Observatory}
\label{chap:pierre-auger-observatory}

The \PAO is the (by area) largest scientific experiment in the world. It 
consists of an array of 1660 \WCDs, which form the \SD, and 27 fluorescence 
telescopes, that make up the \FD.

With a region spanning roughly \SI{3000}{\kilo\meter\squared} it offers a unique
possibility to observe \UHECRs at the tail-end of the \CR energy spectrum. 

We begin this chapter in \cref{sec:science-case} by stating the design 
intentions and formulating open questions that the \PAO aims to answer. Design
details for the \FD and for the \SD are given in \cref{sec:fd} and \cref{sec:sd} 
respectively. After a discussion on the local \DAQ process and the centralized 
event detection in \cref{sec:cdas}, we finish by detailing the procedure of the
event reconstruction and higher level analysis in \cref{sec:rec}.

\section{Science Goal and Open Questions}
\label{sec:science-case}

The flux of cosmic rays above energies $>\SI{e18}{\eV}$ is very low, on average
only 

\subsection{Mass Composition}
\subsection{Muon Deficit}
\subsection{AugerPrime}

\section{The Fluoresence Detector}
\label{sec:fd}



\section{The Surface Detector}
\label{sec:sd}



\section{Central Data Acquisition System}
\label{sec:cdas}



\section{\Offline and Event Reconstruction}
\label{sec:rec}

