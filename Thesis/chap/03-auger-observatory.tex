%! TEX root = ../main.tex

\chapter{The Pierre Auger Observatory}
\label{chap:pierre-auger-observatory}

The \PAO is the (by area) largest scientific experiment in the world. It 
consists of an array of 1660 \WCDs, which form the \SD, and 27 fluorescence 
telescopes, that make up the \FD.

With a region spanning roughly \SI{3000}{\kilo\meter\squared} it offers a unique
possibility to observe \UHECRs at the tail-end of the \CR energy spectrum with
an unprecedented accuracy and precision.

We begin this chapter in \cref{sec:science-case} by formulating open questions 
that the \PAO aims to answer. Design details for the \FD and for the \SD are 
given in \cref{sec:fd} and \cref{sec:sd} respectively. After a discussion of the
local \DAQ processes and the centralized event detection in \cref{sec:cdas}, we 
finish by detailing the procedure of the event reconstruction and higher level 
analysis in \cref{sec:rec}.

\section{Science Goal and Open Questions}
\label{sec:science-case}

The flux of cosmic rays with energies exceeding the ankle, $\SI{5e18}{\eV}$, is 
very low, and measures on average 6 events per \SI{}{\km\squared\year}
\cite{Fenu2023}. It is evident that one needs a large detector and a lot of time
in order to make statistically relevant statements about the physics of \UHECRs.
Altough only one of the initially planned two data taking sites \cite[for PAO 
white paper see][]{Zavrtanik2000} came to reality, the Pierre Auger observatory 
has been be a world-leading experiment in terms of measured exposure from the 
beginning of \DAQ in January 2004 \cite{Abraham2004}, and will continue to yield
results until decomission after 2030 \cite{Castellina2023}.

Many insights, such as the existence of the \CR dipole discussed in
\cref{sec:cr-origins}, have been gathered from Augers event database as a 
consequence. Still, a plethora of mysteries remain. It follows a list of, in no 
particular order, important missing links of information that motivate not least
this thesis, but the continued effort and daily work done by the Auger 
collaboration.

\subsection{Flux supression at highest energies}
\label{ssec:flux-supression}

\subsection{Validity of shower simulations}


\subsection{Exotic air shower events}
\subsubsection{Photon showers}
\subsubsection{Neutrino showers}
\subsubsection{GZ effect}

\section{The Fluorescence Detector}
\label{sec:fd}

The Fluorescence Detector of the \PAO is a set of 27 reflector telescopes tuned
to detect faint sources of \UV light. More specifically, the aim of the \FD is
to observe \UV-emission of \EAS. However, since the solar irradiance 
(\SI{120}{\watt\per\meter\squared} @
\SIrange[range-phrase={--}]{200}{400}{\nano\meter} \cite{Snow2013}) and even the
lunar irradiance (\SI{16}{\nano\watt\per\meter\squared} @
\SIrange[range-phrase={--}]{180}{300}{\nano\meter} \cite{Lean1989}) in the 
\UV-band far outshine the emission of \UV-light by cosmic rays 
(\SI{0.32}{\nano\watt\per\meter\squared} @ \SI{337}{\nano\meter}, for details 
see \cref{app:cr-uv-irradiance}), the \FD can only operate in the astronomical 
night during third to first quarter moon. This consequently drops the duty cycle 
to approximately 13\% \cite{Abraham2010}.

Ignoring three exceptions (see \autoref{ssec:heat}), all telescopes are grouped
at four \FD sites, where a collection of six identical setups offer a $180^\circ
\times30^\circ$ view (Azimuth $\times$ Elevation) over the \SD array. 
\autoref{fig:auger-map} shows the location of these sites relative to the \SD.

\todo{auger map}
\begin{figure}[t]
  \centering
  \subfloat[]{\includegraphics[width=0.48\textwidth]{auger-observatory/auger_array-small-gs.jpg}
  \label{fig:auger-map}
  }\hspace{0.2cm}
  \subfloat[]{\includegraphics[width=0.48\textwidth]{}
  \label{fig:LABEL}
  }
  \caption[]{\subref{fig:auger-map} Overview of the \PAO. The black dots give the 
  location of \WCDs. The blue lines indicate the location and FOV of the \FD. 
  HEAT (red) overlooks the Infill area of the \SD-array \subref{fig:LABEL} }
  \label{fig:pao-images}
\end{figure}



\subsection{Telescope design}
\label{ssec:fd-design}

The \FD components located on the optical axis are (in the order that light 
traverses them):

\begin{itemize}
  \item \textbf{Shutter:} \\
  Consists of two massive metal doors that block the light propagation. Several 
  mechanisms automatically close the shutters to shield the telescope from e.g.
  atmospheric influences like wind and rain, or an unacceptably high light flux.
  \item \textbf{UV Filter:} \\
  Visible background light is irrelevant to the phenomena observed by the \FD
  and can, in some cases, damage electronics. A MUG6 UV-filter is installed in 
  the light path for this reason. It separates the climate controlled inside of 
  the \FD buildings from the outside.
  \item \textbf{Curtain:} \\
  A secondary failsafe, that prevents high-intensity light from reaching the 
  sensitive electronics. During nominal operation the curtain sits outside of
  the light path. It can be deployed in emergency situations, or during daytime
  work and maintenance.
  \item \textbf{Mirror:} \\
  asdasd
  \item \textbf{Camera:} see next subsection.
\end{itemize}

A schematic overview of the setup 

\subsection{Camera design}


\subsection{HEAT}
\label{ssec:heat}



\subsection{Calibration of measurements}
\label{ssec:fd-calibration}

\subsubsection{Drum calibration}
\subsubsection{XY-scanner}

\section{The Surface Detector}
\label{sec:sd}

\todo[inline]{roughly mention design, duty cycle}

\section{Central Data Acquisition System}
\label{sec:cdas}



\section{\Offline and Event Reconstruction}
\label{sec:rec}

\begin{figure}[t]
  \centering
  \subfloat[]{\includegraphics[width=0.47\textwidth]{cosmic-rays/spectrum-double1.png}
  \label{fig:LABEL}
  }\hspace{0.2cm}
  \subfloat[]{\includegraphics[width=0.47\textwidth]{cosmic-rays/spectrum-double2.png}
  \label{fig:LABEL}
  }
  \caption[]{\subref{fig:LABEL} asdasd \subref{fig:LABEL} asdasd}
  \label{fig:}
\end{figure}

\begin{figure}[t]
  \centering
  \includegraphics[width=0.9\textwidth]{cosmic-rays/spectrum-single.png}
  \caption{asdasdasdasd}
  \label{fig:asdasd}
\end{figure}


