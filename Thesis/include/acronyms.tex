\let\oldbaselinestretch=\baselinestretch%
\renewcommand{\baselinestretch}{1.5}%
\large\normalsize%

\pagestyle{plain}

\chapter*{Acronyms}
\label{chapter:acronyms}

This is a list of alphabetically sorted acronyms used throughout this work.

\begin{acronym}[UHECR]

  \setlength{\itemsep}{-\parsep}
  \word{CR}{Cosmic Ray}
  \word{DAQ}{Data Acquisition}
  \word{EAS}{Extensive Air Showers}
  \word{FD}{Fluorescence Detector}
  \word{GAP}{Giant Array Project}
  \word{SD}{Surface Detector}
  \word{PAO}{Pierre Auger Observatory}
  \word{UHECR}{Ultra High Energy Cosmic Ray}
  \word{UV}{ultra violet}
  \word{WCD}{Water Cherenkov Detector}
  % \acro{infillinfill}[SD-433]{\SI{433}{\m} SD infill}
  % \acro{AERA}{Auger Engineering Radio Array}
  % \acro{AGNs}{active galactic nuclei}
  % \acro{AMIGA}{Auger Muon Detectors for the Infill Ground Array}
  % \acro{AoP}{area-over-peak}
  % \acro{ASCII}{Auger Scintillator for Composition - II}
  % \acro{CIC}{Constant Intensity Cut}
  % \acro{cdf}[\cdf]{cumulative distribution function}
  % \acro{CMB}{cosmic microwave background radiation}
  % \acro{EAS}{extensive air shower}
  % \acro{FADC}{flash analog to digital converter}
  % \acro{FoV}{field of view}
  % \acro{GRB}{gamma-ray burst}
  % \acro{GPS}{Global Positioning System}
  % \acro{HEAT}{High Elevation Auger Telescopes}
  % \acro{infill}[SD-750]{\SI{750}{\m} SD vertical}
  % \acro{ICRC}{International Cosmic Ray Conference}
  % \acro{LDF}{lateral distribution function}
  % \acro{LHC}{Large Hadron Collider}
  % \acro{LSD}{Layered surface detector}
  % \acro{MD}{Muon detector}
  % \acro{MoPS}{Multiplicity of positive steps}
  % \acro{pdf}[\pdf]{probability density function}
  % \acro{PE}{photo-electron}
  % \acro{PMT}{photo-multiplier tube}
  % \acro{RD}{Radio detector}
  % \acro{sd}[SD-1500]{\SI{1500}{\m} SD vertical}
  % \acro{SNR}{supernova remnant}
  % \acro{TA}{Telescope Array}
  % \acro{ToT}{time-over-threshold trigger}
  % \acro{ToTd}{time-over-threshold deconvoluted trigger}
  % \acro{VAOD}{vertical aerosol optical depth}
  % \acro{VCT}{vertical centered through-going}
  % \acro{VEM}{vertical-equivalent muon}
  % \acro{WCD}[WCD]{\WCD}

\end{acronym}

\renewcommand{\baselinestretch}{\oldbaselinestretch}%
\large\normalsize%

\newcommand{\PAO}{\ac{PAO}\xspace}
\newcommand{\WCDs}{\acp{WCD}\xspace}
\newcommand{\WCD}{\ac{WCD}\xspace}
\newcommand{\SD}{\ac{SD}\xspace}
\newcommand{\FD}{\ac{FD}\xspace}
\newcommand{\CRs}{\acp{CR}\xspace}
\newcommand{\CR}{\ac{CR}\xspace}
\newcommand{\UHECRs}{\acp{UHECR}\xspace}
\newcommand{\UHECR}{\ac{UHECR}\xspace}
\newcommand{\DAQ}{\ac{DAQ}\xspace}
\newcommand{\GAP}{\ac{GAP}\xspace}
\newcommand{\UV}{\ac{UV}\xspace}
\newcommand{\EAS}{\ac{EAS}\xspace}
\newcommand{\Xmax}{$\text{X}_\text{max}$\xspace}

% \newcommand{\ropt}{r_\mathrm{opt}\xspace}
% \newcommand{\rscal}{r_\mathrm{scale}\xspace}
% \newcommand{\sigshsh}{\sigma_\mathrm{sh-sh}\xspace}
% \newcommand{\fnkg}{f_\mathrm{NKG}(r)\xspace}
% \newcommand{\fnkgbeta}{f_{\mathrm{NKG},\beta}(r)\xspace}
% \newcommand{\InSize}{S_{450}\xspace}
% \newcommand{\lgInSize}{\lg S_{450}\xspace}
% \newcommand{\SdSize}{S_{1000}\xspace}
% \newcommand{\lgSdSize}{\lg S_{1000}\xspace}
% \newcommand{\lgS}{\lg S\xspace}
% \newcommand{\InEE}{S_{35}\xspace}
% \newcommand{\SdEE}{S_{38}\xspace}
% \newcommand{\lgInEE}{\lg S_{35}\xspace}
% \newcommand{\lgSdEE}{\lg S_{38}\xspace}
% \newcommand{\lgE}{\lg E\xspace}
% \newcommand{\lgEeV}{\lg ( E / \si{\eV} )\xspace}
% \newcommand{\ndof}{n_\mathrm{dof}\xspace}
% \newcommand{\fcic}{f_\mathrm{CIC}\xspace}
% \newcommand{\fcicref}[1]{f_\mathrm{CIC}^\mathrm{ref}(#1)\xspace}
% \newcommand{\cossqtheta}{\cos^2 \theta\xspace}
% \newcommand{\lgr}{\lg (r / \si{\m})\xspace}
% \newcommand{\dlgE}{\d{\,\lg E}\xspace}
% \newcommand{\AvgAoP}{\langle\mathrm{AoP}\rangle\xspace}

% \newcommand{\MC}{Monte Carlo\xspace}
% \newcommand{\chisqfit}{chi-squared fit\xspace}
% \newcommand{\chisqfits}{chi-squared fits\xspace}
% \newcommand{\mlfit}{maximum-likelihood fit\xspace}
% \newcommand{\chired}{\chi^2/\ndof}

% \newcommand{\pdf}{p.d.f.\xspace}
% \newcommand{\cdf}{c.d.f.\xspace}

% \newcommand{\defeq}{\vcentcolon=}
% \newcommand{\eqdef}{=\vcentcolon}