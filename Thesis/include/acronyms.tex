\let\oldbaselinestretch=\baselinestretch%
\renewcommand{\baselinestretch}{1.15}%
\large\normalsize%

\chapter*{Acronyms}
\label{chapter:acronyms}

This is a list of alphabetically sorted acronyms used throughout this work.

\begin{acronym}[YTM]

  \setlength{\itemsep}{-\parsep}
  \acro{infillinfill}[SD-433]{\SI{433}{\m} SD infill}
  \acro{AERA}{Auger Engineering Radio Array}
  \acro{AGNs}{active galactic nuclei}
  \acro{AMIGA}{Auger Muon Detectors for the Infill Ground Array}
  \acro{AoP}{area-over-peak}
  \acro{ASCII}{Auger Scintillator for Composition - II}
  \acro{Auger}{Pierre Auger Observatory}
  \acro{CIC}{Constant Intensity Cut}
  \acro{cdf}[\cdf]{cumulative distribution function}
  \acro{CMB}{cosmic microwave background radiation}
  \acro{CR}{cosmic ray}
  \acro{EAS}{extensive air shower}
  \acro{FADC}{flash analog to digital converter}
  \acro{FD}{Fluorescence detector}
  \acro{FoV}{field of view}
  \acro{GRB}{gamma-ray burst}
  \acro{GPS}{Global Positioning System}
  \acro{HEAT}{High Elevation Auger Telescopes}
  \acro{infill}[SD-750]{\SI{750}{\m} SD vertical}
  \acro{ICRC}{International Cosmic Ray Conference}
  \acro{LDF}{lateral distribution function}
  \acro{LHC}{Large Hadron Collider}
  \acro{LSD}{Layered surface detector}
  \acro{MD}{Muon detector}
  \acro{MoPS}{Multiplicity of positive steps}
  \acro{PAO}{Pierre Auger Observatory}
  \acro{pdf}[\pdf]{probability density function}
  \acro{PE}{photo-electron}
  \acro{PMT}{photo-multiplier tube}
  \acro{RD}{Radio detector}
  \acro{sd}[SD-1500]{\SI{1500}{\m} SD vertical}
  \acro{SD}{Surface detector}
  \acro{SNR}{supernova remnant}
  \acro{TA}{Telescope Array}
  \acro{ToT}{time-over-threshold trigger}
  \acro{ToTd}{time-over-threshold deconvoluted trigger}
  \acro{UHECR}{ultra-high energy cosmic ray}
  \acro{VAOD}{vertical aerosol optical depth}
  \acro{VCT}{vertical centered through-going}
  \acro{VEM}{vertical-equivalent muon}
  \acro{WCD}[WCD]{\WCD}

\end{acronym}

\renewcommand{\baselinestretch}{\oldbaselinestretch}%
\large\normalsize%
