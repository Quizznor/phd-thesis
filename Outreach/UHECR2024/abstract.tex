\documentclass{scrartcl}

\usepackage{amsmath}
\usepackage{hyperref}
\usepackage[affil-it]{authblk}

\title{Calibration of the Scintillator Surface Detector of AugerPrime}
\author[1]{Paul Filip}
\author[2]{on behalf of the Pierre Auger Observatory}
\affil[1]{Karlsruhe Institute for Technology, Germany}
\affil[2]{Malargüe, Mendoza Province, Argentina}
\renewcommand\Authand{, }
\date{}

\begin{document}
\maketitle

The Pierre Auger Observatory is a hybrid detector designed to study cosmic rays of the highest energies.
With the area of the detector array covering $3000\,\text{km}^2$ and with more than 20 years of runtime, it counts among the cosmic ray experiments with the highest accumulated exposure worldwide.
The Surface Detector (SD) consists of over 1600 autonomously operating Water-Cherenkov-Detectors(WCDs), distributed on a triangular grid with $1500\,\text{m}$ spacing. Since some years, the Pierre Auger Collaboration has pushed to enhance the composition sensitivity of the SD with installation of additional detectors, such as a radio antenna, and plastic scintillators that are mounted on top of the WCDs. With these new detection channels, Auger will become much more sensitive to the mass of hadronic primaries, and be able to test hadronic interaction models with unparalleled precision. 

We present the two methods used to calibrate the surface scintillator detectors at the Pierre Auger Observatory. While both methods rely on measuring the properties of muons that impinge on the detector, they will be performed independent of one another, and achieve different goals.
The first method analyzes the distribution of pulse-heights recorded from muons hitting the scintillators, and is used offline to calibrate data gathered from extensive air shower events.
The second method is currently under development, and will use a rate-based approach to estimate calibration constants online, for every station autonomously. This will ultimately enable event detection via the scintillator detector only, which will greatly enhance the sensitivity of Auger to the electromagnetic component of extensive air showers.


\end{document}
