\documentclass{scdpg}
\begin{document}
\scBookLanguage{de}
\begin{scAbstract}
\scLanguage{en}
\scTitle{Method and first results of the XY-Scanner Calibration of the Fluorescence Detector of the Pierre Auger Observatory}
\scAuthor{*}{Paul}{Filip}{1}
\scAuthor{}{Christoph}{Schäfer}{1}
\scCollaborationName{Pierre-Auger-}
\scAffiliation{1}{Hermann-von-Helmholtz-Platz 1 76344 Eggenstein-Leopoldshafen}
\scBeginText
The Pierre Auger Observatory is a hybrid instrument designed to detect extensive air showers stemming from ultra-high-energy cosmic rays impinging on the upper atmosphere of the earth. It uses two independent methods of detection. The Fluorescence Detector (FD) observes the evolution of the air shower in the atmosphere, and provides a model independent estimation of the energy of a cosmic ray primary particle. Additionally, data gathered from the FD is used to calibrate the energy scale of the Surface Detector (SD) array, which measures the shower footprint on the ground.

In this talk, we present a novel method of calibration for the FD, which relies on a UV-light source mounted on a motorized XY-stage. The light source exposes the telescope camera sensor to light pulses of known intensity. The presented setup simplifies the calibration procedure drastically and is able to improve the systematic uncertainty of the FD calibration from $\sim9\%$ to $\sim4.4\%$. In addition, the short- and long term stability of the procedure is analyzed using data from the seven measurement campaigns.
\scEndText
\scConference{Karlsruhe 2024}
\scPart{T}
\scContributionType{Vortrag;Talk}
\scTopic{4.07 Experimentelle Techniken der Astroteilchenphysik; 4.07 Experimental Techniques in Astroparticle Physics}
\scKeywords{Auger; Fluoresence; Detector; Calibration}
\scEmail{paul.filip@kit.edu}
\scCountry{Germany}
\end{scAbstract}
\end{document}
