\documentclass{scdpg}
\begin{document}
\scBookLanguage{de}
\begin{scAbstract}
\scNoUseTeX
\scLanguage{en}
\scTitle{The XY-Scanner of the Pierre Auger Observatory}
\scAuthor{*}{Paul}{Filip}{1}
\scAffiliation{1}{Hermann-von-Helmholtz-Platz 1 76344 Eggenstein-Leopoldshafen}
\scBeginText
The Pierre Auger Observatory is a hybrid detector designed to detect extensive air showers stemming from ultra-high-energy cosmic rays (UHECRs) impinging on the upper atmosphere of the earth. It uses two independant methods of detection. The surface detector (SD) consists of ~1600 water tanks, who have a 100% duty cycle, but whose energy scale rely on model-dependant Monte-Carlo simulations. The Fluoresence Detector (FD) on the other hand, while being limited to an uptime of ~15%, offers a model independant estimation of the energy of a cosmic ray primary particle.

In this talk, we present a novel method of calibration for the FD, which relies on a UV-lightsource on a motorized XY-stage. The presented setup simplifies calibration procedure drastically, and is able to improve the absolute uncertainty of the FD calibration from ~9% previously to ~4.4%
\scEndText
\scConference{Karlsruhe 2024}
\scPart{T}
\scContributionType{Vortrag;Talk}
\scTopic{4.07 Experimentelle Techniken der Astroteilchenphysik; 4.07 Experimental Techniques in Astroparticle Physics}
\scKeywords{Auger; Fluoresence; Detector; Calibration}
\scEmail{paul.filip@kit.edu}
\scCountry{Germany}
\end{scAbstract}
\end{document}
