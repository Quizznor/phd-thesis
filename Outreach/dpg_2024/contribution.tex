\documentclass{scdpg}
\begin{document}
\scBookLanguage{de}
\begin{scAbstract}
\scLanguage{en}
\scTitle{Method and first results of the XY-Scanner of the Pierre Auger Observatory}
\scAuthor{*}{Paul}{Filip}{1}
\scAuthor{}{Christoph}{Schäfer}{1}
\scAffiliation{1}{Hermann-von-Helmholtz-Platz 1 76344 Eggenstein-Leopoldshafen}
\scBeginText
The Pierre Auger Observatory is a hybrid detector designed to detect extensive air showers stemming from ultra-high-energy cosmic rays (UHECRs) impinging on the upper atmosphere of the earth. It uses two independant methods of detection. The Fluoresence Detector (FD) observes the evolution of the air shower in the atmosphere, and provides a model independant estimation of the energy of a cosmic ray primary particle. Additionally, data gathered from the FD is used to cross-calibrate the energy scale of the Surface Detector (SD), a set of $\sim1600$ water thanks that measure the shower footprint on the surface.

In this talk, we present a novel method of calibration for the FD, which relies on a UV-lightsource mounted on a motorized XY-stage. The lightsource exposes the telescope camera sensor to light pulses of known intensity. The presented setup simplifies the calibration procedure drastically, and is able to improve the systematic uncertainty of the FD calibration from $\sim9\%$ to $\sim4.4\%$. In addition, the short- and longterm stability of the procedure is analyzed using data from the first seven measurement campaigns.
\scEndText
\scConference{Karlsruhe 2024}
\scPart{T}
\scContributionType{Vortrag;Talk}
\scTopic{4.07 Experimentelle Techniken der Astroteilchenphysik; 4.07 Experimental Techniques in Astroparticle Physics}
\scKeywords{Auger; Fluoresence; Detector; Calibration}
\scEmail{paul.filip@kit.edu}
\scCountry{Germany}
\end{scAbstract}
\end{document}
